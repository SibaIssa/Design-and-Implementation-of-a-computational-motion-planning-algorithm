تعد مضمنات هذا البحث أحد القضايا التي تطرق لها الباحثون بكثرة واهتمام في الفترة الأخيرة: التشكيل السرابي للروبوتات غير الهولنومية.  هناك العديد من الطرق التي قد توصل مجموعة من الروبوتات من منطقة الى أخرى في فضاء العمل بسلام (أحياناً)، لكن عديدها إما تحوي تعقيدات برمجية وبنية خوارزمية ومخطط حالة غاية في التعقيد، أو أنها لا توفر حل في أغلب الأحوال. نحاول في هذا البحث تقديم خوارزمية تخطيط مسار لسرب من الروبوتات الغير هولنومية في بيئة مكتظة باعتبارها تابعة بالحالة للحل الرقمي لمعادلة جريان المائع من منطقة إلى الأخرى بشكله المستقر.
من مزايا الحل المقدم أنه رخيص حسابياً ويمتلك العديد من العناصر القابلة للتسريع باستخدام أدوات حسابية وعتاد حاسوبي جديد، لكن محور البحث يكمن في مكان آخر. نحاول في هذا البحث تقديم طريقة حتمية لإيجاد مسارات حركة هذه الآليات من منطقة لأخرى في فضاء العمل، فالمعادلات الرياضية لجريان السوائل حتماً ستجد حلاً يقوم بإيصال جزيئات المائع من المنطلق للهدف. هذه الحتمية ستعطي أفضلية واضحة للطريقة المقترحة لسببين: موثوقية الطريقة (إذا كان الوصول ممكن فهو حتماً ضمن مجموعة الحل)، ونعومة المسارات المنتجة (مما سيمكن الآليات الغير هولنومية من تقليص الفارق العتادي مع الهولنومية).
سيتم اختبار طريقة الحل المقترحة على منصة روبوتات قمنا بتصميمها وتنفيذها بشكل كامل، مع مراعاة عوامل مثل جودة التصنيع والمرونة والسهولة في الاستخدام والتعديل والتكلفة المناسبة. تم تصميم المنصة بحيث أن يتم التحكم والتواصل مع الروبوتات بواسطة حاسوب مركزي ويتم مراقبتها بواسطة كاميرا مُثبتة أعلى المنصة.
