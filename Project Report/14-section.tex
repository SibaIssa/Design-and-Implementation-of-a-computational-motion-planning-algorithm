تخدم العجلات وظيفتين رئيسيتين: الأولى هي دورها الأساسي لتحريك الروبوت، والثانية هي كأقراص مشققة لتسجيل حركة المحركات. نناقش في هذا القسم الأدوات الرياضية والشروط التصميمية الموافقة لما سبق.
العجلات للحركة
عند اجراء التجارب المخبرية على سرعات المحركات الموجودة متبوعة بعلب سرعتها، وجدنا أن السرعة عند $ 7.5 V $ (أي السرعة الأعظمية) كانت $ 135 $ دورة في الدقيقة. عند وضع شرط أن الروبوت يجب أن يكون قادر على قطع الحلبة شاقولياً أو افقياً خلال مدة أقصاها عشر ثوان، وجدنا أن السرعة الخطية يجب أن تكون أكبر من $ 0.24\ m.s^{-1} $. هذا جعل من السهل حساب أن نصف قطر الدولاب المطلوب يجب أن يكون أكبر من $ 1.7E-2\ m.  $. ان اختيار نصف قطر الدولاب أيضاً يخضع أيضاً لقيد آخر هو شقوق مسجلات الحركة.



\subsubsection{العجلات كمسجلات حركة}

من المعلوم أنه لإنشاء نظام تحكم دقيق يجب وجود نظام تحسس للزاوية وسرعة حركة المحركات. أول طريقة وأكثرها شيوعاً هي استخدام قرص مشقق بفتحات ذات ابعاد متساوية. يقوم مقطع ضوئي بمقاطعة المتحكم عند تسجيل مرور فتحة أمامه مما يسمح للمتحكم بتسجيل عبور هذه الفتحة، مما يؤدي لتسجيل تغير في الزاوية مقداره الزاوية المصممة بين الفتحتين$  \vartheta $.

\begin{equation}
	\vartheta=\frac{2\pi}{n}
\end{equation}

حيث يمثل n عدد الفتحات الموجودة على القرص. نعرف $ \phi $ بأنها الزاوية التي يمسحها المقطع الضوئي لعبور شق واحد. ومنه يمكن تعريف $ \eta=\frac{\phi}{\vartheta}$ ، وهي نسبة عرض الشق الى مجمل الزاوية $ \vartheta $. يمكن مضاعفة الدقة عن طريق جعل وحدة المقاطعة الخارجية الموجودة في المتحكم AVR تقوم بتسجيل الحافة الصاعدة والهابطة في إشارة المقطع. فتصبح $  \Delta d $ (أصغر مسافة يمكن للمسجل قراءتها):

\begin{equation}
\Delta d=\frac{\vartheta}{2}r=\ \frac{\pi}{n}r
\end{equation}

يسمح وجود مقطع ضوئي واحد بتسجيل تغيرات الزاوية دون معرفة اتجاه الدوران. لقراءة اتجاه الدوران يجب إضافة مقطع ضوئي ثان متموضع مع فرق طور مقداره $ \vartheta/4  $ بالنسبة للمقطع الأول. ان تحريك المقطع الضوئي الثاني بعدد صحيح من مضاعفات $ \vartheta $ لن يغير من أداء عمله، لكن سيمكن المصمم من وضع كلا الحساسين في مكان مريح للفك والتركيب السريعين، مما يتوافق مع أحد أهم شروط الهيكل الخارجي للروبوت. نسمي المسافة بين الحساسين $  \psi $، ونعتبر أن الحساس الأول موضوع في المبدأ. 

بنقل هذه الشروط الى معادلات رياضية:

\begin{equation}
\vartheta=\frac{2\pi}{n}
\end{equation}

\begin{equation}
	\rightarrow\psi=\frac{\vartheta}{4}+{k}\vartheta=\left(\frac{\eta}{2}+{k}\right)\vartheta
\end{equation}

حيث $ {k}\in\mathbb{Z} $. إذا وضعنا شروط التالية:

\begin{itemize}
	\item دقة تسجيل الحركة يجب أن تكون أصغر من الدقة التي نستطيع الحصول عليها من الكاميرا.
	\item 	الزاوية بين المقطعين ($ \psi $) يجب أن تكون بين $ \frac{\pi}{4} $ و $ \frac{3\pi}{4} $.
	\item 	الزاوية بين المقطعين يجب أن تكون عدد صحيح من مضاعفات $ \frac{\pi}{180}\times5 $ لتسهيل عملية التصميم.
	\item 	عرض الشق لا يقل عن $1.2 E-3m$.
\end{itemize}

تولد الشروط مسألة البرمجة الصحيحة الآتية:

إيجاد $k, \eta, n, r$ حيث:
\begin{equation}
\psi=\frac{\phi}{2}+\mathcal{k}\vartheta=\left(\frac{\eta}{2}+\mathcal{k}\right)\vartheta
\end{equation}
\begin{equation}
\Delta d<2.5E-3\rightarrow\frac{r}{n}<2\times3.979E-4
\end{equation}
\begin{equation}
\psi=\mathcal{q}\ \left(\frac{\pi}{180}\times5\right),\ where\ \mathcal{q}\in\mathbb{Z}
\end{equation}
\begin{equation}
\frac{\pi}{4}\le\psi\le\frac{3\pi}{4}
\end{equation}


تسمى المسألة السابقة بأنها مسألة برمجة صحيحة بسبب وجود شرط انتماء $k$ الى مجموعة الأعداد الصحيحة. تم اثبات ان مسائل البرمجة الصحيحة هي مسائل NP-Hard ويحتاج حلها الى استخدام احدى تقنيات البحث مع القليل من الأمثلة. لن نتطرق الى استخدام هذه الطرق بسبب قلة عدد القيود على المسألة وصغر فضاء الحل، اذ يمكن عبور كل مجال الحل وتسجيل أفضل نقطة.


بعد حل المسألة السابقة يمكن استخلاص ما يلي:
\begin{equation}
\eta=0.5
\end{equation}
\begin{equation}
r=2E-2\ m
\end{equation}
\begin{equation}
\psi=210\times\frac{\pi}{180}\ rad
\end{equation}
\begin{equation}
n=33
\end{equation}
\begin{equation}
\Delta d=1.9E-3\ m
\end{equation}
ومنه ينتج تصميم العجل الموضح في الشّكل \ref{14:fig:1}.

\begin{figure}[htbp]
	\centering
	\includesvg[width=0.7\linewidth]{figs/14/fig1}
	\caption{تصميم عجلات الروبوت الناتج}
	\label{14:fig:1}
\end{figure}

