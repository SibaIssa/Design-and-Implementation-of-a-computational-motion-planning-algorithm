كُتب برنامج تشغيل الروبوت باستخدام لغة c لقربها من العتاد الصلب ولدعمها من قبل منظومة GCC. كون المعالج بمعمارية 8 بت، هناك العديد من القيود على سرعة تنفيذ العمليات الحسابية عليه، فكلفة حساب متحول صحيح 32 بت أضعاف كلفة حساب ذات الرقم اذا كان مخزناً في متحول 8 بت. لا تواجه المعالجات الحاسوبية ذات معمارية 64 بت هذه المشاكل بشكل عام.

يمكن القول بأن نقطة الضعف الرئيسية لهذه المعالجات تكمن في عرض مسجلاتهاـ حيث تتميز بطيف واسع من المزايا المناسبة بشكل مطلق للتطبيق الحالي. 


صمم نظام تشغيل الروبوت بطريقة البرمجة التابعية لخلو لغة c من الأغراض والصفوف. نجد في الملحق توصيف شامل لكل أجزاء الكود البرمجي.

يتبع الروبوت المسار الواصل إليه عن طريق مصحح متقطع PID. تتكون حلقة الحساب من مصححين: الأول يتحكم بسرعة المحركين لعدم المسافة بين مركز الروبوت والنقطة الهدف، والثاني يقوم بعدم الزاوية بي شعاع سرعة الروبوت والشعاع الواصل بين مركز الروبوت والنقطة الهدف. عمل هذين المصححين ينتج مسارات مناسبة بين النقاط المتتابعة.

