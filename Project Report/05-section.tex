إنّ مفهوم الأسراب مأخوذ من الطّبيعة، حيث تمّت محاكاة أسلوب التّعاون بين الأسراب البيولوجيّة (النّمل، النّحل، الأسماك..) ونُقل هذا المفهوم إلى الرّوبوتات. يمكننا تصنيف الأسراب البيولوجيّة إلى أسراب أرضيّة وأخرى جويّة، وكذلك الأمر بالنّسبة للرّوبوتات. يختلف هذان النّوعان فيما بينهما من ناحية التّطبيق بشكل أساسي وقد اعتمد البحث \cite{b1} على تصنيف التطبيقات حسب نوع المهمّة الّتي تتولّاها الرّوبوتات (تغطية منطقة ما، مهمّات خطيرة، مهمّات يلعب الوقت فيها دور أساسي...).
يعدّ تخطيط المسار الحسابي للروبوتات الأرضيّة مشكلة مطروقة جداً في الوقت الحالي لما لها من أهمية في توفير الوقت، الطاقة، والعتاد الحسابي. حسابياً، إنّ مشكلة تخطيط المسار الأمثل لأي آلية هي مسألة \textenglish{NP-Hard} \cite{b2,b3}. تتعقد أليات الحساب وتبعاته عند التعامل مع بيئة ديناميكية، أو عند القيام بتخطيط مسار أكثر من كائن في ذات الوقت.
يهدف المشروع الى البحث في إمكانية تطوير آليات لتخطيط المسار الحسابي لمجموعة كائنات اعتماداً على معادلات الجريان النّاتجة عن حل المعادلات التّفاضلية الجزئيّة لحركة الموائع. لقد قدّم كلّ من نافيير وستوكس معادلات كافية لنمذجة حركة السائل في الظروف المُختلفة، وسيتمّ الاعتماد عليها في إنشاء المسارات.
