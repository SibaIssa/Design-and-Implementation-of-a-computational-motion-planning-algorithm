أفضت الطريقة المتبعة في انتاج مسار لروبوت واحد في فضاء اختياري نتائج عالية الوثوقية والأمان، التجارب أثبتت بشكل كبير قدرة هذه الطريقة على إنتاج طريق ناعم حال وجود اتصال بين المبدأ والهدف. يعد تعميم هذه الطريقة لضمان وصول مجموعة روبوتات إلى هدفها عملية غاية في التعقيد، وذلك خصوصاً لطبيعة الروبوتات الغير هولنومية المدروسة في هذا المشروع. يمكن الاستدلال إلى تهجين بين عدة طرق لإيجاد ما يناسب حالات خاصة من فضاءات اعمل في الدراسات المرجعية.
لنتمكن من انتاج حل يتسم بالبساطة الاستدلالية وبالحتمية والكمالية، يجب العودة إلى الآلية التي تتصرف فيها جزيئات سائل وجدت نفسها في حقل كموني. إن ألية التفكير هذه تفضي إلى نوعين من الحلول:

\begin{itemize}
	\item الروبوتات غير واعية لبعضها
	\item الروبوتات (الجزيئات) واعية لبعضها
\end{itemize}

يمكن مقاربة الحل الأول بجزيئات سائل لا نيوتوني تسير في حقل السرعة المنتج، والطريقة الثانية بعدد من الأجسام صلبة تهوي في هذا الحقل.

\subsubsection{الروبوتات غير الواعية لوجود السرب}

هنا يتم احتساب حقل كموني صغير خارج من كل روبوت في السرب وإضافة هذه الحقول إلى حقل السرعة منهاً للتصادم، يتم هنا احتساب جميع قيم السرعة في كل نقاط الفضاء عن طريق أخذ جريان المائع من جهة والجيران من جهة أخرة. يمكن التحكم بقوة الحقل الكموني النافر الصادر عن كل روبوت بتغيير بارامترات الحساب.
ألية الحساب:

يتم أولاً ملء الخريطة فارغة بأجسام تتناسب أبعادها مع أبعاد الروبوتات الباقية في السرب، ثم يتم حساب تحويل المسافة للمصفوفة الناتجة \textenglish{(distance transform).} يتبع هذه الخطوة القيام بعملية مشروحة في الجدول \ref{11:fig:process}.





	\begin{table}
		\centering
		\begin{tabular}{cp{150pt}}
			\begin{subfigure}{0.35\textwidth}
				\centering
				\includesvg[width=0.9\linewidth]{figs/11/fig5}
			\end{subfigure}&  انشاء مصفوفة بأبعاد الخريطة ياوضع عليها الروبوتات الباقية من السرب. أبعاد الروبوتات في المصفوفة يتناسب مع أبعاد الروبوت (المصفوفة $\mathcal{M}$) \\
			\begin{subfigure}{0.35\textwidth}
				\centering
				\includesvg[width=0.9\linewidth]{figs/11/fig6}
			\end{subfigure}& التحويل المسافي $D(\mathcal{M})$\\
			\begin{subfigure}{0.35\textwidth}
				\centering
				\includesvg[width=0.9\linewidth]{figs/11/fig7}
			\end{subfigure}&  $\mathcal{M}_2 = \epsilon (1/D(\mathcal{M}) - 1/k )^2$ \\
			\begin{subfigure}{0.35\textwidth}
				\centering
				\includesvg[width=0.9\linewidth]{figs/11/fig8}
			\end{subfigure}& $ \mathcal{M}_2 > k $ \\
			\begin{subfigure}{0.35\textwidth}
				\centering
				\includesvg[width=0.9\linewidth]{figs/11/fig9}
			\end{subfigure}& $ \nabla (\mathcal{M}_2 > k) $
		\end{tabular}
		\caption{عملية حساب المسار في الحالة الأولى من الحل}
		\label{11:fig:process}
	\end{table}


ثم يتم جمع الحقل الناتج إلى حقول السرعة وإنتاج الحقل الخاص بكل روبوت. انزلاق الروبوت على الحقل الناتج يضمن وصوله إلى الهدف بسلاسة. انظر الشكل \ref{11:fig:square_map} كمثال.

\begin{figure}[htbp]
	\centering
	\includesvg[width=0.7\linewidth]{figs/11/square_map}
	\caption{مثال عن الطريقة المقترحة}
	\label{11:fig:square_map}
\end{figure}

\subsubsection{الروبوتات الواعية لوجود السرب}


يتم في هذه الطريقة برمجة وجود نابض ومخمد بين كل روبوتين متجاورين من روبوتات السرب. وجود هذه النظام الميكانيكي سيضمن حركة السرب ضمن تشكيلة معينة نجمية تحدد عناصرها من بارامترات البرمجة. يمكن حساب القوة المتبادلة بين روبوتين متجاورين كالتالي:

\begin{equation}
F_{ij}=F_k+F_c=k\left(d_{ij}-d_0\right)+c\frac{d}{dt}\left(d_{ij}-d_0\right)
\end{equation}
وبعدها يتم تحصيل القوى باستخدام:
\begin{equation}
\vec{\mathcal{F}_i}=\ \sum_{j\in\mathbb{N}_i}{\vec{F_{ij}}\ }
\end{equation}


تجمع القوة السابقة مع القوة الناتجة عن وجود حقل سرعة مائع لإنتاج خطوة المشي الجديدة. انظر الشكل \ref{11:fig:6}

\begin{figure}[htbp]
	\centering
	\includesvg[width=0.7\linewidth]{figs/11/fig61}
	\caption{القوى المتبادلة بين الروبوتات ممثلة كعناصر ميكانيكية}
	\label{11:fig:6}
\end{figure}